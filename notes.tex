\documentclass[12pt, a4paper, english]{article}
\usepackage[T1]{fontenc}
\usepackage[utf8]{inputenc}
\usepackage{babel, graphicx, textcomp, varioref, hyperref}
\usepackage{geometry, marginnote, listings, courier}
\usepackage[dvipsnames]{xcolor}

\definecolor{light-gray}{gray}{0.95}

\lstset{ %
  backgroundcolor=\color{light-gray},   % choose the background color; you must add \usepackage{color} or \usepackage{xcolor}; should come as last argument
  basicstyle=\footnotesize\ttfamily,        % the size of the fonts that are used for the code
  breakatwhitespace=false,         % sets if automatic breaks should only happen at whitespace
  breaklines=true,                 % sets automatic line breaking
  captionpos=b,                    % sets the caption-position to bottom
  commentstyle=\color{LimeGreen},    % comment style
  deletekeywords={...},            % if you want to delete keywords from the given language
  escapeinside={\%*}{*)},          % if you want to add LaTeX within your code
  extendedchars=true,              % lets you use non-ASCII characters; for 8-bits encodings only, does not work with UTF-8
  frame=none,	                   % adds a frame around the code
  keepspaces=true,                 % keeps spaces in text, useful for keeping indentation of code (possibly needs columns=flexible)
  keywordstyle=\color{blue},       % keyword style
  language=c++,                 % the language of the code
  morekeywords={*,...},           % if you want to add more keywords to the set
  %numbers=left,                    % where to put the line-numbers; possible values are (none, left, right)
  %numbersep=5pt,                   % how far the line-numbers are from the code
  %numberstyle=\tiny\color{gray}, % the style that is used for the line-numbers
  rulecolor=\color{black},         % if not set, the frame-color may be changed on line-breaks within not-black text (e.g. comments (green here))
  showspaces=false,                % show spaces everywhere adding particular underscores; it overrides 'showstringspaces'
  showstringspaces=false,          % underline spaces within strings only
  showtabs=false,                  % show tabs within strings adding particular underscores
  %stepnumber=2,                    % the step between two line-numbers. If it's 1, each line will be numbered
  stringstyle=\color{BrickRed},     % string literal style
  tabsize=2,	                   % sets default tabsize to 2 spaces
  title=\lstname                   % show the filename of files included with \lstinputlisting; also try caption instead of title
}

\hypersetup {
  colorlinks = true,
  linkcolor = blue,
}

\title{Notes on ROS}
\author{Bård-Kristian Krohg \\ Universitetet i Oslo \\
\texttt{baardkrk@ifi.uio.no}}%\and My Collaborator}

\pagestyle{headings}

\begin{document}

\maketitle{}

%\begin{lstlisting}
%  #include <stdio.h>
%
%  int
%  main(int argc, char *argv)
%  {
%    // This is just a small test..
%    /* Are this comment style also recognized? */
%    std::cout << "Hello World!";
%  }
%\end{lstlisting}

\section{Introduction}
In this document I intend to keep some preliminary notes on using ROS.

\section{Cheat sheet}

\subsection{General}
\texttt{rospack} - get information about packages. Use for example \\
\texttt{rospack find [package\_name]} \\
\texttt{roscd} - change directory to package or other things \\
\texttt{roscd [location[/subdir]]} \\
\texttt{rosls} - works just as \texttt{ls}. \\
\texttt{rosnode} - can be used with for example \texttt{list} or \texttt{info}. \\
\texttt{rosrun} - Use as \texttt{rosrun [package\_name] [node\_name]} \\
\texttt{rostopic} - Overwiew of ros topics, we can again use \texttt{list} and so on. We can also publish information to topics by using
\begin{lstlisting}
rostopic pub [topic] [msg_type] [args]
\end{lstlisting}
We can also publish a set number of times using the option \texttt{-\$NUMBER} \\
\texttt{rqt} - nice package to see information about the system. plot, graph and so on.
\subsection{Creating packages}
Remember to source your setup file, \texttt{\$ source \textasciitilde/catkin\_ws/devel/setup.bash}
To create a package, use the script (while in your catkin workspace)
\begin{lstlisting}
  catkin_create_pakage [pakage_name] [depend1] [depend2] [depend3]
\end{lstlisting}
Go to \texttt{catkin\_ws} and do \texttt{catkin\_make}. Then, edit the \texttt{package.xml} file.
\subsubsection{Configure CMakeLists.txt}
You \emph{MUST} follow the format below, and the order in the configuration counts:
\begin{enumerate}
  \item Required CMake Version \\
  \texttt{cmake\_minimum\_required(VERSION 2.8.3)}\\
  All packages require a minimum version of 2.8.3 or higher.
  \item Package Name \\
  \texttt{project(name)} \\
  You can reference this name anywhere later in the CMake script by using the variable \texttt{\$\{PROJECT\_NAME\}}
  \item Find other CMake/Catkin packages needed for build \\
  \texttt{find\_package(catkin REQUIRED COMPONENTS [package1] [package2])}\\
  Using the \texttt{COMPNENTS} option in \texttt{find\_package} appends include paths, libraries, etc. that is needed for the packages, to the \texttt{catkin\_varables}.
  \item Enable Python module support \\
  \texttt{catkin\_python\_setup()} \\
  Call this if your project uses python modules.
  \item Message/Service/Action Generators \\
  \texttt{add\_message\_files(), add\_service\_files(), add\_action\_files()}
  \item Invoke message/service/action generation \\
  \texttt{generate\_messages()}
  \item Specify package build info export \\
  \texttt{catkin\_package()}\\
  Has 5 optional arguments
  \begin{itemize}
    \item \texttt{INCLUDE\_DIRS} - the exported include paths (i.e. cflags) for the package.
    \item \texttt{LIBRARIES} - The exported libraries from the project.
    \item \texttt{CATKIN\_DEPENDS} - Other catkin projects that this project depends on.
    \item \texttt{DEPENDS} - \emph{Non}-catkin CMake projects that this project depends on.
    \item \texttt{CFG\_EXTRAS} - Additional configuration options
  \end{itemize}
  \item Libraries/Executables to build \\
  \texttt{add\_library() / add\_executable() / target\_link\_libraries()}
  \item Tests to build \\
  \texttt{catkin\_add\_gtest()}
  \item Install rules \\
  \texttt{install()}
\end{enumerate}

\subsection{Services and Parameters}
\subsubsection{\texttt{rosservice}}
Services are another way (apart form topics) that ros nodes can communicate with each other. Services allow nodes to send a \textbf{request} and receive a \textbf{response}.
Can use services with \texttt{rosservice call}.
\subsubsection{\texttt{rosparam}}
Manipulating data on the ROS Parameter Server. The data is stored using YAML, where 1 is integer, 1.0 is float and so on.

\end{document}

\documentclass[UKenglish,twoside,12pt]{report}     %% ... or USenglish or norsk or nynorsk
\usepackage[utf8]{inputenc}         %% ... or utf8 or applemac
\usepackage[T1]{fontenc,url}
\urlstyle{sf}
\usepackage{babel,textcomp,csquotes,duomasterforside,varioref,graphicx,acro,hyperref}
\usepackage[backend=biber,style=numeric-comp]{biblatex}

\usepackage[inner=3.5cm, outer=2.5cm, a4paper]{geometry} %% if not enough materials, change outer to 4.5
%\usepackage{showframe}
\usepackage{lipsum}

\title{Modelling Human Behavior and State}        %% ... or whatever
\subtitle{A machine learning approach}         %% ... if any
\author{Bård-Kristian Krohg}                      %% ... or whoever 

\addbibresource{bib/sources.bib}                  %% ... or whatever
%%\addbibresource{sources.bib}
%%\bibstyle{plain}
%%% ------- front, main and backmatter
\makeatletter

\newcommand\frontmatter{%
    \cleardoublepage
  %\@mainmatterfalse
  \pagenumbering{roman}}

\newcommand\mainmatter{%
    \cleardoublepage
 % \@mainmattertrue
  \pagenumbering{arabic}}

\newcommand\backmatter{%
  \if@openright
    \cleardoublepage
  \else
    \clearpage
  \fi
 % \@mainmatterfalse
   }

\makeatother


\begin{document}
\duoforside[dept={Institute for informatics},
  program={Informatics: Robotics and Intelligent Systems},
  long,
  printer={X-press printing house},
  image={img/vitruvian.png}]{}
%% 悟 - enlightenment, percieve
%% 徴 - indication, sign, symptom
\frontmatter
\maketitle{}

\chapter*{Abstract}

Short intro to the project (1/2 pages)

\begin{itemize}
\item what is it about (problem)
\item what has been done to ready the problem (method, data)
\item findings (main findings)
\item precautions for the findings
\item conclusion
\item implications
\end{itemize}


                   %% ... or Sammendrag or Samandrag
\chapter*{Preface}                    %% ... or Forord

TODO: When / Where / COINMAC / Kyushu University


Tell me, have you heard the story of \emph{Darth Pelagius the Wise}? I thought not. It is not a story the \emph{Jedi} would tell you.
Ironic, he had the power to save others, but not himself.

\section*{Acknowledgements}

I would like to thank Professor Jim Torresen and Vice Dean Ryo Kurazume for the oppurtunity to write this masters thesis at the lab here in Kyushu.
TODO: friends, family~\cite{hindawi2016case}

\tableofcontents{}
\listoffigures{}
\listoftables{}
% probably a good idea for the nomenclature entries:
\acsetup{first-style=short}

\DeclareAcronym{ros}{
  short = ROS ,
  long = Robotic Operating System ,
  class = abbrev
}
\DeclareAcronym{hr}{
  short = HR ,
  long = heart rate ,
  class = abbrev
}
\DeclareAcronym{rr}{
  short = RR ,
  long = respiration rate ,
  class = abbrev
}


% class `abbrev': abbreviations:
\DeclareAcronym{ny}{
  short = NY ,
  long  = New York ,
  class = abbrev
}
\DeclareAcronym{la}{
  short = LA ,
  long  = Los Angeles ,
  class = abbrev
}
\DeclareAcronym{un}{
  short = UN ,
  long  = United Nations ,
  class = abbrev
}

% class `nomencl': nomenclature
\DeclareAcronym{angelsperarea}{
  short = \ensuremath{a} ,
  long  = The number of angels per unit area ,
  sort  = a ,
  class = nomencl
}
\DeclareAcronym{numofangels}{
  short = \ensuremath{N} ,
  long  = The number of angels per needle point ,
  sort  = N ,
  class = nomencl
}
\DeclareAcronym{areaofneedle}{
  short = \ensuremath{A} ,
  long  = The area of the needle point ,
  sort  = A ,
  class = nomencl
}

\printacronyms[include-classes=abbrev, name=Abbrevations]{}
\printacronyms[include-classes=nomencl, name=Nomenclature]{}

\mainmatter
%%\part{Introduction}                   %% ... or Innledning or Innleiing

\chapter{Background}
Some background on the project
\section{Elderly Care}
\subsection{Facilities}
Elderly people are often put in homes to better get help with their needs.
\subsection{Living at Home}
Home nursery

\section{Multimodal Elderly Care System}
A system to assist elderly living at home. 


\chapter{Earlier work}
%% \section{lorem}
%% \lipsum
%% \section{ipsum}
%% \lipsum
%% \subsection{dolor sit amet}
%% \lipsum

%%\part{The project}                    %% ... or ??

\chapter{Planning the project}        %% ... or ??


%%\part{Conclusion}                     %% ... or Konklusjon
\chapter{Method}
USE A DESCRIPRIVE TITLE! (ie. Robot Brain Implementation)
See your choises in birds eye perspective. how you can discuss what youve done.

What was made, why is it good. document all choices. The rule is that a different scientist with the same resources should be able to get the same results based on your description. Dont write about the methods in general, but exactly what youre doing. discuss your own approach. Show that youre aware there are other alternatives to what you did, and what advantages a different method may have. Did you have any influence over how things were solved. what is strengths and weaknesses about your work. what would you do different if you were to do it again? be open about weaknesses, but defend your choices. Refer to other researchers who has done the same. 
\chapter{Experiments}                     %% ... or ??
What did we find out. dont overcomplicate the explanation. this could be the longest part of the thesis. about 15-20 pages? If you have more questions, use that as structure for this section. You can divide this into multiple chapters: subsidiary questions to the main theme, hypothesies, themes. One to three chapters are usually OK. the most important first, main findings. small neuances exceptions and discussions. discuss what youve found. this could be a chapter in itself.
\section{Results}
the main finings, as simply put as possible.
\section{Discussion}
Look at the results critically, weaknesses to the method. discuss how the findings can be explained. reasons to why you found what you found. how it compares to earier research. can reference some earlier theory.
\chapter{Conclusion}
About 10\% of the lenght (means \~8 pages)
often the only thing that is read by people who are just looking at the thesis.
\begin{itemize}
\item tell in short version what youve found. main findings first. short, simply put. the neuances and details can be fleshed out in the following sections.
\item how your findings fit with earlier work and research. (dont repeat too much from the ``earlier research'' or ``theory'' chapters. ) What fits, and suggestions as to why.
\item The way your finings can have significance. Can we see the subject in a new way? should one change something in practice or how one does things because of your research? can the finds benefit society. Youre going to tell the world, and see what youre writing about in a bigger picture. Can other people learn something from this?
\item OpenPose~\cite{cao2017realtime}
\end{itemize}
\chapter{Future Work}
What research is missing, what do we want to know more about, what other methods should be tried out

\chapter{Preliminary Notes and sources}
\section{Human Pose}
\href{https://autostudentsite.wordpress.com/2017/05/18/running-and-building-nite2-samples-for-kinect-v2/}{How to run NiTE2 on linux for comparison with windows software:}

\subsection{papers}

\href{https://arxiv.org/pdf/1611.08050.pdf}{Open Pose paper}

\href{https://arxiv.org/pdf/1701.07372.pdf}{Multi view RGB-D approach for pose estimation}

\href{https://arxiv.org/pdf/1601.01006.pdf}{Space-time representation of people based on 3d skeletal data}

\href{https://arxiv.org/pdf/1603.06937.pdf}{Stacked hourglass networks for human pose estimation}

\href{https://arxiv.org/pdf/1705.03098.pdf}{Baseline for human 3d pose estimation from 2d images}
\href{https://www.youtube.com/watch?v=Hmi3Pd9x1BE&feature=youtu.be}{Accompanying video}
\href{https://github.com/una-dinosauria/3d-pose-baseline}{Github repo}

\href{https://arxiv.org/pdf/1607.02046.pdf}{Mocap guided data augmentation for 3d pose estimation in the wild}

\href{http://human-pose.mpi-inf.mpg.de/contents/andriluka14cvpr.pdf}{2D human pose estimation}

\href{https://www.sciencedirect.com/science/article/pii/0734189X85900945}{Determination of 3D human body postures from a single view}

\href{http://journals.sagepub.com/doi/full/10.1177/1729881416657746}{People detection and tracking using RGBD cameras for mobile robots}
\href{http://journals.sagepub.com/doi/pdf/10.1177/1729881416657746}{Paper direct link}

\href{http://media.cs.tsinghua.edu.cn/~imagevision/papers/\%5B2016\%5D0000266-HuZhan-ICIP2016.pdf}{Fast Human Detection in RGB-D Images based on color depth joint feature learning +RoI Extraction}

\href{https://hal.inria.fr/inria-00590212/file/3dpvt-skeleton.pdf}{3D skeleton-based body pose Recovery}

\href{https://arxiv.org/pdf/1705.01583.pdf}{VNect real time 3D human pose estimation with single rgb camera}
\href{http://gvv.mpi-inf.mpg.de/projects/VNect/}{Project page}

\href{http://www.reactivereality.com/static/pdf/paper69.pdf}{Skeletal graph based human pose estimation in real-time}

\href{https://www.sciencedirect.com/science/article/pii/S026288561100134X}{Human skeleton tracking from depth data using geodesic distances and optical flow}

\href{http://citeseerx.ist.psu.edu/viewdoc/download?doi=10.1.1.642.3647&rep=rep1&type=pdf}{Multi-modal Surface Registration for Markerless Initial Patient Setup in Radiation Therapy using Microsoft’s Kinect Sensor}

\href{https://ieeexplore.ieee.org/stamp/stamp.jsp?tp=&arnumber=6126310&tag=1}{Accurate 3D pose Estimation from a Single Depth Image}

\href{http://www.mva-org.jp/Proceedings/2015USB/papers/14-18.pdf}{3D Hand Skeleton Model Estimation from a Depth Image}

\href{https://www.microsoft.com/en-us/research/wp-content/uploads/2016/02/ks_book_2012.pdf}{Key Developments in Human Pose Estimation for Kinect}

\href{https://arxiv.org/pdf/1712.03453.pdf}{Single-Shot Multi-Person 3D body pose estimation from monocular rgb input}

\href{https://arxiv.org/pdf/1612.06524.pdf}{3D human pose estimation = 2D pose Estimation + Matching}

\href{https://hal.inria.fr/hal-01505085/document}{LCR-Net: Localization-Classification-Regression for Human Pose}

\href{https://arxiv.org/pdf/1802.04216.pdf}{Image-based Synthesis for Deep 3D Human Pose Estimation}

\href{http://www.cs.toronto.edu/~jtaylor/papers/cvpr2012.pdf}{The Vitruvian Manifold: Inferring Dense Correspondences for One-Shot Human Pose Estimation}



\subsubsection{Skeleton Fitting}

\href{https://pdfs.semanticscholar.org/8492/075b6d9ed4a3065849d0b0eb7a705a5112b9.pdf}{Skeleton Fitting Techniques for optical motion capture}

\href{http://vision.gel.ulaval.ca/~vignolaj/vignolaVI03.pdf}{Progressive Human Skeleton Fitting}

\href{https://link.springer.com/content/pdf/10.1007/978-3-642-15193-4_45.pdf}{Learning Inverse Kinematics for Pose-Constraint Bi-Manual Movements}

\href{https://pdfs.semanticscholar.org/8492/075b6d9ed4a3065849d0b0eb7a705a5112b9.pdf}{Local and Global Skeleton FItting Techniques for Optical Motion Capture}

\href{https://link.springer.com/chapter/10.1007\%2F978-3-642-25489-5_16}{3D Body Pose Estimation Using an Adaptive Person Model for Articulated ICP}
\href{https://link.springer.com/content/pdf/10.1007\%2F978-3-642-25489-5_16.pdf}{Paper}

\subsubsection{Head tracking}
\href{https://arxiv.org/pdf/1309.3418.pdf}{Face direction using depth maps}

\href{http://www.dgcv.nii.ac.jp/Publications/Papers/2005/elcviaVol5No3-05.pdf}{Detecting Human Heads with their orientation}

\href{https://arxiv.org/abs/1611.10195}{POSEidon Face From Depth for Driver pose Estimation}

\href{https://www.youtube.com/watch?v=JO8XHzc6JPQ}{Face Tracking in OpenCV}

\subsection{datasets}
\href{http://vision.imar.ro/human3.6m/description.php}{Human3.6m dataset for pose}
\href{http://tele-immersion.citris-uc.org/berkeley_mhad}{Berkeley MHAD (Multimodal Human Action Database)}



\section{Vital detection and mood}
\subsection{papers}

\href{https://www.ncbi.nlm.nih.gov/pmc/articles/PMC5579477/table/sensors-17-01776-t002/}{Heart Rate detection using Microsoft Kinect}
\href{https://www.ncbi.nlm.nih.gov/pmc/articles/PMC5579477/}{Full text}

\href{https://blogs.msdn.microsoft.com/kinectforwindows/2015/06/12/detecting-heart-rate-with-kinect/}{Detecting Heart Rate with Kinect v2}
\href{https://github.com/dngoins/Kinectv2HeartRate}{Github repo}

\subsection{datasets}
\href{http://www.socsci.ru.nl:8180/RaFD2/RaFD?p=main}{Radboud Faces Database (Emotions)}



\section{Human Activity recognition}
\subsection{papers}

\href{https://www.hindawi.com/journals/cin/2016/4351435/}{Human Activity Recognition system using skeleton data from rgbd sensors}

\href{https://www.hindawi.com/journals/jr/2017/7610417/}{Tracking a Subset of Skeleton Joints: An Effective Approach towards Complex Human Activity Recognition}

\subsection{datasets}

\href{https://github.com/jysung/activity_detection}{Human Activity Detection project at Personal Robotics Lab at Cornell University github repo}
\href{http://pr.cs.cornell.edu/humanactivities/data.php#format}{Dataset by this lab}
\href{http://pr.cs.cornell.edu/humanactivities/results.php}{Results}


\section{Other}
\href{https://github.com/code-iai/iai_kinect2}{IAI Kinect2}

\href{https://github.com/facebookresearch/Detectron}{Facebook's Detectron github repo}

\subsection{papers}
\href{https://arxiv.org/pdf/1406.2661.pdf}{Generative Adversarial Nets (GANs)}

(\href{https://www.int-arch-photogramm-remote-sens-spatial-inf-sci.net/XL-1/301/2014/isprsarchives-XL-1-301-2014.pdf}{RGB-D Indoor Plane based 3d modeling using autonomous robot})

\href{https://arxiv.org/pdf/1707.03770.pdf}{Q learning}

\href{https://mdanderson.influuent.utsystem.edu/en/publications/real-time-range-imaging-in-health-care-a-survey}{Real Time range imaging in health care A survey}
\href{https://link.springer.com/content/pdf/10.1007\%2F978-3-642-44964-2_11.pdf}{pdf:}

\subsection{datasets}

\href{http://www.vision.ee.ethz.ch/en/datasets/}{ETHZurich CVL datasets}

\href{http://www.michaelfirman.co.uk/RGBDdatasets/}{List of RGBD datasets}

\href{http://www.tlc.dii.univpm.it/blog/databases4kinect}{Databases 4 kinect}


\backmatter
\printbibliography

\clearpage
\newpage
\chapter*{Notes}

\end{document}

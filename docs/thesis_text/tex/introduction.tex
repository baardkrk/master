\chapter{Introduction}

%% Why collecting human pose
%% General introduction to this project

%% Themes addressed in the master thesis:
% Deep Learning
% Object recognition
% Human Pose
% Depth images
% Human Robot interaction
% Human Activity Recogntion
% Elderly Care

%% Fundamental requirements and goals:
% Recognizing lying-down pose
% Ensuring privacy
% Work on limited hardware
% Real time

%% Most important contributions of the work:
As life expectancy increases in Norway, so does the population who needs geriatric care either at home, or in a geriatric facility. Accodring to a communal health sruvey \cite{oslohelsa}, it is projected that the increasing senior population in Oslo will lead to increased pressure on the healthcare services. To mitigate the need for help at home, it is important to encourage health-promoting activities, and uncover what and where preventative measures is needed. The \gls{mecs} is proposed as a solution where data gathered in users homes, guide where and what help is needed from healthcare personel -- be it preventative or acute.

The \gls{mecs} is envisioned as a small, mobile unit which can be introduced to any home. This eliminates the user from the operational loop of the unit, making the data gathering process robust to forgetfulness or physical ability of the user. One of the key information points gathered by the \gls{mecs} will be the users physical pose. The users physical pose informs the system of the folowing:\\
\textbf{Body language}, enabling the possibility of smooth \gls{hri} when moving around, or determining the intent of the user.\\
\textbf{Physical state}, informing the \gls{mecs} wether the user is in need of acute help.\\
\textbf{History of natural posture}, which could be invaluable information for a physical therapist or doctor to develop personalized training programs preventing muscle degradation.\\
\textbf{Activity recognition}, helping the \gls{mecs} decide on what actions to execute. For example, reminding the user to take their prescribed medication if they forget.

All these key points could be captured by a 2D system, however a 2D representation of a pose would not be robust to different viewing angles. A 3D representation could define the origin of two poses' coordinate system to the same landmark in each pose, making it is easy to compare the two poses by for example the euclidian distance between the poses' corresponding landmarks.

Estimating human pose in 3D, or \gls{mocap} is a well known area of research. However, \gls{mocap} is expensive, and requires a large amount of physical hardware. The industry standard is to use an elaborate \gls{mocap} studio that requires multiple expensive cameras, a large area, and specialized software. Therefore the application areas for motion capture are currently mostly limited to research, movie-, and videogame-making. The \gls{mocap} problem deals with finding a representation of an actor, creature, or object that can be used in animation. This representation is often a \emph{rigged} skeleton with bones that define the movement of the animated character \cite{skeletonAnimation}. The movements of the computerized rig is mapped to the movements of an observed actor in the real world and recorded.

%% However, the problem \gls{mocap} technology solves, estimating human pose, have many application areas: analyzing movements, activity recognition, importing natural motions to \gls{vr}, and improving \gls{hri}. 

%% A low-cost alternative to an expensive \gls{mocap} rig could be a single depth camera. With such a low-cost entry point, human pose estimation could be used for remote, or automatic, feedback for \emph{how} a motion should be executed. This could be useful for physical therapists or other instructors doing remote work. In addition, traditional dances or ceremonies could be digitized in a way that preserves precise movement patterns and coordination between multiple persons.

%% In a setup where multiple depth cameras are synchronized, the observed area could also be much larger than would be possible in a traditional \gls{mocap} studio.

\section{Research Goals}

This work explores the possibility of using a single depth camera to solve the \gls{mocap} problem of estimating human pose in 3D. To be viable in a mobile unit with limited processing capability, the resulting method needs to be lightweight and fast enough to process human motions. The resulting method must also be accurate enough for healthcare personel or the \gls{mecs} itself to extract reliable and useful information about the pose.

The method presented in this thesis will therefore be evaluated using the following goals:

%% The detected human pose is represented as a rig as seen in figure \ref{fig:skeleton_markers}. 
%% A convolutional neural network with a stacked, novel articulation network is introduced as a way to find the poses in the depth images.

%% Extra emphasis has been placed on keeping the networks as small as possible to keep computation cost down on low-spec hardware while making sure the pose estimation can still run in real-time.

\begin{enumerate}
\item Propose a lightweight system for recognizing human pose in 3D based on depth images. 
\item Design and develop an architecture which could be deployed to systems with limited hardware, and provide useful information.
\item Evaluate the preformance of the system and find room for improvements for it.
\end{enumerate}

\section{Contributions}

The main contribution of this work is the proposal of using a shallow \gls{cnn} trained on depth images, in combination with a novel articulation network to define human pose in 3D. The \gls{cnn}s main task is to propose a set of poses present in the depth image. These poses are refined in the articulation network, which allows for a shallower \gls{cnn} architecture. This leads then to fewer parameters in the system, which leads to a more lightweight method that should preform faster on limited hardware.

\section{Thesis Structure}

%% An algorithm using sequential refinement-steps with a combination of bottom-up evidence gathering, and a top-down neural network for articulation estimation is explored. The work also explores how shallow the network can be, as well as other optimization techniques for running on mid- to low-end hardware.

%% The algorithm will be able to

%% Keypoints
%% + 3D 
%% + Realtime
%% + Depth images
%% + Limited hardware

%% PAF 2D vector field -> 3D vector field
%% Crucial for accuracy

%% Articulation network for occluded keypoints

%% Publicly available ROS package

 
%% \section{Multimodal Elderly Care System}
%% As life expectancy increases in Norway, so does the population who needs geriatric care either at home, or in a geriatric facility. Accodring to \cite{oslohelsa}, it is projected that especially the elderly population in Oslo will increase in the coming years, and that this will lead to increased pressure on the healthcare services. This means that it will be even more important to encourage health-promoting, and take preventative measures.

%% As part of the effort to let people live independently at home for as long as possible, the \gls{mecs} was proposed. One of the goals for the \gls{mecs} is to function as an autonomous safety alarm, a device that lets elderly living at home call the emergency services at the click of a button. However, if the emergency is an accident which renders the user incapacitated, or otherwise unconscious, an alarm that requires interaction will not be of much help. In contrast, the \gls{mecs} can monitor a user, and warn healthcare personnel in case of an occuring, or predicted, emergency. The \gls{mecs} will even be able to send contextual information to first-responders, which lets them better prepare for the situation.

%% The system is specified to be non-invasive, as this will increase the convenience for the user, in that the \gls{mecs} will not require any interaction from the user to function. Taking the user out of the operational loop has other advantages as well. For example, a monitoring system in the form of a smartwatch, will be inconvenient and ineffective the user forgets to put it on, whereas a passive system will not need the user to preform any action to be effective.

%% Further, it is hypothesized that the information gathered by the \gls{mecs} can help doctors or physical therapists prescribe or recommend health-promoting activities for the user. This could help prevent accidents or lifestyle diseases -- which again will help relieve pressure on the communal healthcare services.

%% \section{Hardware}
%% The \gls{mecs} is envisioned to be a small, mobile unit as this can be introduced to any home without extensive alterations to the environment, thus lowering the cost of the system and reusability of the units. As the \gls{mecs} would need a charging station anyway, we propose a master/slave configuration between a stationary and a mobile unit, which should communicate through a secure wireless connection, for example a WLAN. We let the stationary unit take care of the power consuming complex processing, extending the operational time for the mobile unit between battery charges. We will therefore assume the system has access to mid- to high-end personal GPU/processing hardware, when we evaluate the real world practicality/runtime of the algorithm.

%% To provide as good a service for the user as possible, we believe that gathering many channels of information will be helpful. We wish to learn the users daily activity patterns or vital signs so when unhealthy or risk-filled patterns emerge, preventative actions can be implemented. \gls{har}, gate/mood recognition, or detection of vital signs all require us to know where the user is in the scene.

%% This also places some requirements on our system. If the system solely relies on this work to find humans in the scene, we set our lower framerate limit to 8 fps to be able to preform human heart rate aquisition~\cite{Wu12Eulerian}\footnote{If the maximum human heart rate is 220 bpm, and we want to measure it accuratley using video sequences produced by the \gls{mecs}, our sampling frequency needs to be higher than $7.\overline{3}$ Hz in order to satisfy the Nyquist rate.}. Further, the \gls{mecs} needs to be able to recognize humans in unstructured environments, in a variety of poses and to diffrentiate between multiple people.

%% In order to log the users activity patterns, and detect anomalies or deviations in this, we envision the \gls{mecs} doing \gls{har}.

%% \section{Privacy}
%% %% TODO: first draft -- rewrite!!!
%% The information gathered by this system should only be available to the users designated doctors/physisians and to the user themselves, and should be treated at the same classification level as any persons medical journal. This means that the dataprocessing from the \gls{mecs} must happen on-site, or that any cloud processing happens under a user agreement that protects this data from those owning the servers. The same agreement should count for anyone making the hardware/software that is used in the \gls{mecs}.

%% The \gls{mecs} is also capable of gathering additional data that is not relevant for the healthcare personell. For example, by using depth sensors we are able to create 3d maps of the users home or environment. this is helpful for the \gls{mecs}, however it is not relevant for the healthcare personel. (with the exception of first responders, which could get the information either through a descriptive message -- ``the patient is in the bathroom on the second floor, no elevator, steep stairs'' or via the actual internal map the \gls{mecs} has created for its own internal navigation.)


\chapter{Introduction}

\section{Multimodal Elderly Care System}
As life expectancy increases in Norway, so does the population who needs geriatric care
either at home, or in a specialized geriatric facility. Accodring to \cite{oslohelsa}, it
is projected that mainly the elderly population in Oslo will increase in the coming years,
and that this will lead to increased pressure on the healthcare services. \gls{cnn}s are
widely used.

As part of the effort to let people live independently at home for as long as possible, we propose a \gls{mecs}. One of the goals for the \gls{mecs} is to function as an autonomous safety alarm, a device that lets elderly living at home call for help in an emergency. However, if the emergency is an accident which renders the user incapacitated, or otherwise unconscious, the manual safety alarms will not be of much help. In contrast the \gls{mecs} can monitor a user, and warn healthcare personnel in case of an occuring, or predicted, emergency.

We also stribe to make the system non-invasive, as this will increase the convenience for the user, and takes the user out of the loop in terms of running the system. For example, a monitoring system in the form of a smartwatch, will be inconvenient and ineffective the user forgets to put it on.

It is hypothesized that the information gathered by the \gls{mecs} can help doctors or physical therapists to prescribe or recommend health promoting measures/activities for the user. This could help prevent accidents or lifestyle diseases -- which again will help lessen the pressure on the communal healthcare services.

The \gls{mecs} is envisioned as a small mobile unit as this can be introduced to any home without extensive alterations to the environment, thus lowering the cost of the system and reusability of the units. We propose a master/slave configuration between a stationary and a mobile unit, which should communicate through a secure wireless connection, for example a WLAN. We let the stationary unit take care of the power consuming processing, extending the operational time for the mobile unit between battery charges. The stationary unit could also act as a charging station. Therefore, we will assume the system has access to high-end GPU/processing facilities, that would be too cumbersome for a mobile unit, when we evaluate the preformance of this work.

To provide as good a service for the user as possible, we believe that gathering many channels of information will be helpful. We wish to learn the users daily activity patterns or vital signs so when unhealthy or risk-filled patterns emerge, preventative actions can be implemented. \gls{har}, gate/mood recognition, or detection of vital signs all require us to know where the user is in the scene.

This also places some requirements on our system. If the system solely relies on this work to find humans in the scene, we set our lower framerate limit to 8 fps to be able to preform human heart rate aquisition~\cite{Wu12Eulerian}\footnote{If the maximum human heart rate is 220 bpm, and we want to measure it accuratley using video sequences produced by the \gls{mecs}, our sampling frequency needs to be higher than $7.\overline{3}$ Hz in order to satisfy the Nyquist rate.}. Further, the \gls{mecs} needs to be able to recognize humans in unstructured environments, in a variety of poses and to diffrentiate between multiple people.

In order to log the users activity patterns, and detect anomalies or deviations in this, we envision the \gls{mecs} doing \gls{har}. 

\section{Human Pose Estimation}
For The \gls{mecs} to function as intended, it is imperative that it can detect humans reliably. And for \gls{har} we need accurate pose estimation. 

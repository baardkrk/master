\chapter*{Abstract}


This work is part of a larger project where we explore bringing robotics into geriatric care. The goal of this project is to create a robotic system that can assist in optimizing the use of caregivers, so they are used where they are needed.

This work will focus on capturing information about the user, anonymization of the data, what data is necessary or ethical to capture, limitations for on-location data processing and what data can be sent for further processing in the cloud, or human analysis.

We will also implement an ethical data-collection suite for the open-source Robotic Operating System, which can be implemented on a wide variety of robots.

Convolutional Neural Networks have been used for solving object recognition in 2D images with great success. This work aims to use the same techniques to extract 3D human pose from depth images in real-time. We will use two multi-staged CNNs, one to encode the location of each joint, and another to encode the association between the joints to do this.


%% \begin{itemize}
%% \item what is it about (problem)
%% \item what has been done to ready the problem (method, data)
%% \item findings (main findings)
%% \item precautions for the findings
%% \item conclusion
%% \item implications
%% \end{itemize}


%% We look at novel ideas for detecting human pose in three dimensions for use in human activity recognition in geriatric care. We explore using convolutional neural networks directly on depth maps or indirectly, by transfering 2D pose found in complementary rgb images, and extrapolating the 3D joint locations.


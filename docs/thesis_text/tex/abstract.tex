\chapter*{Abstract}


This work is part of a larger project that explores the possibility of bringing robotics into geriatric care. The goal of that project is to create a robotic system that can assist in optimizing the use of caregivers, so they are used where they are needed.

This work introduces \emph{DepthPose} -- a system that solves human pose estimation in 3D for full skeletons. The system is lightweight and can be deployed on mobile units.

Convolutional Neural Networks have been used for solving object recognition in 2D images with great success. This work aims to use the same techniques to extract 3D human poses from depth images in real-time. Two multi-staged CNNs, one to encode the location of each joint and another to encode the association between the joints, provide initial poses and locations for perceived people. The locations are then refined using a novel articulation network.


%% \begin{itemize}
%% \item what is it about (problem)
%% \item what has been done to ready the problem (method, data)
%% \item findings (main findings)
%% \item precautions for the findings
%% \item conclusion
%% \item implications
%% \end{itemize}


%% We look at novel ideas for detecting human pose in three dimensions for use in human activity recognition in geriatric care. We explore using convolutional neural networks directly on depth maps or indirectly, by transfering 2D pose found in complementary rgb images, and extrapolating the 3D joint locations.


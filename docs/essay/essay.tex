\documentclass[a4paper]{report}
\usepackage[utf8]{inputenc}

\title{Human Behavior}
\author{Bård-Kristian Krohg \\ \texttt{baardkrk@student.matnat.uio.no}}

\begin{document}
\maketitle
\section{Abstract}
\section{Motivation}
Our ever aging population\footnote{find sources for this} demands ever more attention and resources from healthcare personell. That, coupled with a shrinking workforce\footnote{find source} raises the demand for technology to effectivize and streamline taking care of elderly.

More people want to stay at home for longer, as elderly homes are often associated with and a reminder that their life is on its last legs. However living at home puts an extra sprain on the systems, as healthcare workers have to spend more time travelling between homes, and response time in case of emergencies will also be slower in comparison with elderly homes with a physitian already on guard\footnote{On guard?! change this}. 

Elderly living at already have the ability to call for help in emergencys through a manual alarm they carry around the neck. Again, travel time for the first responders will be a problem. In addition, many users worry about being responsible for hitting the button themselves.

The Multimodal Elderly Care System aims to create a remote monitoring unit that will help in this process. One of the main goals for this system is to function as an automatic safety alarm for elderly people. Our proposed system consists of a robot equipped with various sensors for remote monitoring of the user. Our focus is safety and ease of use. Therefore we've focused on creating a system that doesn't require the user to do anything special\footnote{Bring anything, charge anything, restructure the environment}. We also wanted a system that didn't require the user to move into a specially built home, or refurbish their current home.

Our solution was therefore to create a mobile robot that will be able to monitor the user in their home with as little restrictions as possible\footnote{we assume neither the robot or the user are able to climb stairs...}.

\section{Background}

\section{Feature selection}
The people in the target group that was intervjued by the design department expressed that they wondered what the system would be able to do for \emph{them}. Naturally, to the neural models to predict abnormal behavior, we need to gather the relevant data first. This implies that the system must be placed in homes, not really operating in an optimal way. Neural network algorithms applied to usecases before they are properly trained can have disaterous results. As an example the YouTube demonetization algorithm was applied while still misclassifying videos resulting in cumberance to the YouTube content creators.

Therefore, while data is not available we've chosen to focus on detecting preconditions which we believe will lead to states in whihch the user might require attention from healthcare personell.

\subsection{Preconditions}
A short description of the different preconditions we want to detect. 

\subsection{Actual features}
Posture -- Mood -- Heart rate -- Respiration rate --  

\section{Implementation}
\subsection{Sensors}
\subsection{Data Gathering}

\end{document}

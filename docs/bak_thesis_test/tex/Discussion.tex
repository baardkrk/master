\section{Discussion}
{\color{red}Look at the results critically, weaknesses to the method. discuss how the findings can be explained. reasons to why you found what you found. how it compares to earier research. can reference some earlier theory.}

A learning based technique akin to \cite{cao2017realtime} where we instead of training the network on annotated 2D joint and limb locations, we train the network on depth maps might yield good results. One can think that the network would be able to learn the rules of anatomy where corresponding limbs should have roughly equal lengths, and the normal human body proportions. This would result in a bottom-up algorithm that won't be slowed down by having multiple people in frame. The guiding runtime factor would mainly be the size of the image being processed.


